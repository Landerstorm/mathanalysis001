% 笔记模板

% 定义文档类型为ctexbook(中文书本)
\documentclass[12pt, a4paper, oneside]{ctexbook}

% 导入各种包
\usepackage[dvipsnames]{xcolor}
\usepackage{tcolorbox}
\usepackage{bm}        % 用于加粗公式中的字母
\usepackage{amsmath}   % 数学公式
\usepackage{amsthm}    % 定理环境
\usepackage{amssymb}   % 更多公式符号
\usepackage{graphicx}  % 插图
\usepackage{mathrsfs}  % 数学字体
\usepackage{enumitem}  % 列表
\usepackage{geometry}  % 页面调整
\usepackage{unicode-math}
\usepackage[colorlinks,linkcolor=black]{hyperref}

% 创建各种环境
\newtheorem{example}{\color{SeaGreen}{例}}  % 整体编号
\newtheorem{algorithm}{算法}
\newtheorem{theorem}{定理}[section]  % 按 section 编号
\newtheorem{definition}{定义}[section]
\newtheorem{axiom}{公理}
\newtheorem{property}{性质}
\newtheorem{proposition}{命题}
\newtheorem{lemma}{引理}
\newtheorem{corollary}{推论}
\newtheorem{remark}{注解}
\newtheorem{condition}{条件}
\newtheorem{conclusion}{结论}
\newtheorem{assumption}{假设}

% 两种方式定义中文的 证明 和 解 的环境:
% 方式一:(缺点:\qedhere 命令将会失效【技术有限,暂时无法解决】)
\renewenvironment{proof}{\par\textbf{证明.}\;}{\qed\par}
\newenvironment{solution}{\par{\textbf{解.}}\;}{\qed\par}
% 方式二:(缺点:\bf 是过时命令,可以用 textb f等替代,但编译会有关于字体的警告,不过不影响使用【技术有限,暂时无法解决】)
%\renewcommand{\proofname}{\indent\bf 证明}
%\newenvironment{solution}{\begin{proof}[\indent\bf 解]}{\end{proof}}


%↓↓↓↓↓↓↓↓↓↓↓==================== 以下是自定义的命令 =================================↓↓↓↓↓↓↓↓↓↓↓

\def\d{\textup{d}} % 直立体 d 用于微分符号 dx
\def\R{\mathbb{R}} % 实数域

% 用于调整表格的高度  使用 \hline\xrowht{25pt}
\newcommand{\xrowht}[2][0]{\addstackgap[.5\dimexpr#2\relax]{\vphantom{#1}}}

% 表格环境内长内容换行
\newcommand{\tabincell}[2]{\begin{tabular}{@{}#1@{}}#2\end{tabular}}

% 使用\linespread{1.5} 之后 cases 环境的行高也会改变,重新定义一个 ca 环境可以自动控制 cases 环境行高
\newenvironment{ca}[1][1]{\linespread{#1} \selectfont \begin{cases}}{\end{cases}}
% 和上面一样
\newenvironment{vx}[1][1]{\linespread{#1} \selectfont \begin{vmatrix}}{\end{vmatrix}}

\def\d{\textup{d}} % 直立体 d 用于微分符号 dx
\def\R{\mathbb{R}} % 实数域
\newcommand{\bs}[1]{\boldsymbol{#1}}    % 加粗,常用于向量
\newcommand{\ora}[1]{\overrightarrow{#1}} % 向量

% 数学 平行 符号
\newcommand{\pll}{\kern 0.56em/\kern -0.8em /\kern 0.56em}

% 用于空行\myspace{1} 表示空一行 填 2 表示空两行  
\newcommand{\myspace}[1]{\par\vspace{#1\baselineskip}}
%↑↑↑↑↑↑↑↑↑↑========================================================================↑↑↑↑↑↑↑↑↑↑


% 页码设置
\geometry{top=25.4mm,bottom=25.4mm,left=20mm,right=20mm,headheight=2.17cm,headsep=4mm,footskip=12mm}

% 设置列表环境的上下间距
\setenumerate[1]{itemsep=5pt,
                partopsep=0pt,
                parsep=\parskip,
                topsep=5pt}
\setitemize[1]{itemsep=5pt,
                partopsep=0pt,
                parsep=\parskip,
                topsep=5pt}
\setdescription{itemsep=5pt,
                partopsep=0pt,
                parsep=\parskip,
                topsep=5pt}


% --------------------------*- ↑ 导言区 ↑ -*--------------------------------
% -------------------------------******-------------------------------------

\title{{\Huge{\textbf{笔记本标题}}}\\——副标题}
\author{Lance}
\date{\today}
\linespread{1.5}

\begin{document}

    \maketitle

    \pagenumbering{roman}   % 页码字体为roman
    \setcounter{page}{1}    % 当前页设置为 page 1

    % 前言
    \begin{center}
        \Huge\textbf{前言}
    \end{center}~\
    这是笔记的前言部分. 
    ~\\
    % 右对齐署名, 日期
    \begin{flushright}
        \begin{tabular}{c}
            Lance\\
            \today
        \end{tabular}
    \end{flushright}

    \newpage
    \pagenumbering{Roman}
    \setcounter{page}{1}

    \tableofcontents

    \newpage
    \setcounter{page}{1}
    \pagenumbering{arabic}


    \chapter{曲线积分与曲面积分}

    在这里可以输入笔记的内容. 

        \section{Gauss 公式}

        这是笔记的正文部分.

        \begin{remark}
            以下是随机选取的演示部分(仅作演示).
        \end{remark}

        % 墨绿色文本框美化
        \begin{tcolorbox}[
            title=\textbf{\textcolor{white}{Theorem}},
            colback=Emerald!10,
            colframe=cyan!40!black]

            \begin{theorem}
                设空间闭区域$\Omega$由分片光滑的闭曲面$\Sigma$围成,若函数$P(x,y,z),\;Q(x,y,z) \\ ,R(x,y,z)$在$\Omega$上具有一阶连续偏导数,则有
                \begin{equation}
                    \iiint_{\Omega} \bigg(\frac{\partial P}{\partial x}+\frac{\partial Q}{\partial y}+\frac{\partial R}{\partial z} \bigg)\d v = \oiint_{\Sigma}P\d y\d z+Q\d z\d x+R\d x\d y
                \end{equation}
                或
                \begin{equation}
                    \iiint_{\Omega} \bigg(\frac{\partial P}{\partial x}+\frac{\partial Q}{\partial y}+\frac{\partial R}{\partial z} \bigg)\d v = \oiint_{\Sigma}(P\cos\alpha+Q\cos\beta+R\cos\gamma)\d S
                \end{equation}
                这里$\Sigma$是$\Omega$的整个边界曲面的外侧,$\cos\alpha,\;\cos\beta,\;\cos\gamma$是$\Sigma$在点$(x,y,z)$处的法向量的单位余弦.
            \end{theorem}

        \end{tcolorbox}

        \begin{proof}
            设闭区域$\Omega$在$xOy$平面上的投影区域为$D_{xy}$,假定穿过$\Omega$内部且平行于$z$轴的直线与$\Omega$的边界曲面$\Sigma$的交点恰是两个,可设$\Sigma$由$\Sigma_1,\;\Sigma_2$和$\Sigma_3$三部分组成,其中$\Sigma_1$和$\Sigma_2$分别由方程$z = z_1(x,y)$和$z = z_2(x,y)$给定,这里$z_1(x,y) \le z_2(x,y)$,$\Sigma_1$取下侧,$\Sigma_2$取上侧,$\Sigma_3$是以$D_{xy}$的边界曲线为准线而母线平行于$z$轴的柱面上的一部分,取外侧.
        
            由三重积分的计算法,有
            \begin{align*}
                \iiint_{\Omega}\frac{\partial R}{\partial z}\d v & = \iint_{D_{xy}}\bigg(\int_{z_1(x,y)}^{z_2(x,y)} \frac{\partial R}{\partial z}\d z\bigg) \d x\d y \\
                                                                 & =\iint_{D_{xy}}\Big\{R\big[x,y,z_2(x,y)\big]-R\big[x,y,z_1(x,y)\big]\Big\}\d x\d y
            \end{align*}
            又由曲面积分的计算法,有
            \begin{gather*}
                \iint_{\Sigma_1}R(x,y,z)\d x\d y = -\iint_{D_{xy}}R\big[x,y,z_1(x,y) \big]\d x\d y \\
                \iint_{\Sigma_2}R(x,y,z)\d x\d y = \iint_{D_{xy}}R\big[x,y,z_2(x,y) \big]\d x\d y
            \end{gather*}
            而$\Sigma_3$上任意一块曲面在$xOy$面上的投影为零,由对坐标的曲面积分的定义可知
            \[
                \iint_{\Sigma_3}R(x,y,z)\d x\d y = 0
            \]
            上述三式相加有
            \[
                \oiint_{\Sigma}R(x,y,z)\d x\d y = \iint_{D_{xy}}\Big[R\big[x,y,z_2(x,y)\big]-R\big[x,y,z_1(x,y)\big]\Big]\d x\d y
            \]
            易得
            \[
                \iiint_{\Omega}\frac{\partial R}{\partial z}\d v = \oiint_{\Sigma}R(x,y,z)\d x\d y
            \]
            如果穿过$\Omega$内部且平行于$x$轴的直线以及平行于$y$轴的直线与$\Omega$的边界曲面$\Sigma$的的交点恰好也是两个,类似地可得
            \[
                \iiint_{\Omega}\frac{\partial P}{\partial x}\d v = \oiint_{\Sigma}P(x,y,z)\d y\d z,\enspace \iiint_{\Omega}\frac{\partial Q}{\partial y}\d v = \oiint_{\Sigma}Q(x,y,z)\d z\d x
            \]
            上述三式相加即有 \textbf{Gauss} 公式.
        \end{proof}

        \begin{example}
            求微分方程$y''-2y'-3y=3x+1$的一个特解.
        \end{example}

        \begin{solution}
            这是二阶常系数非齐次线性微分方程,且函数$f(x)$是$e^{\lambda{x}}P_m(x)$型,其中
            \[
                \lambda = 0,\;P_m(x) = 3x+1
            \]
            与所给方程对应的齐次方程为
            \[
                y''-2y'-3y=0
            \]
            其特征方程为
            \[
                r^2-2r-3 = 0
            \]
            由于$\lambda = 0$不是特征方程的根,所以设特解
            \[
                y* = b_0 x + b_1
            \]
            带入所给方程,得
            \[
                -3b_0 x - 2b_0 - 3b_1 = 3x+1
            \]
            比较等式两端$x$同次幂的系数,易得$b_0 = -1,\;b_1 = \dfrac{1}{3}$,于是求得一个特解为
            \[
                y* = -x + \frac{1}{3}
            \]
        \end{solution}


        % 经典蓝色文本框美化
        \begin{tcolorbox}[
            title=\textbf{\textcolor{white}{Definition}},
            colback=SeaGreen!10!CornflowerBlue!10,
            colframe=RoyalPurple!55!Aquamarine!100!]
            
            \begin{definition}
                设二元函数$f(P) = f(x,\,y)$的定义域为$D$,点$P_0(x_0,\,y_0)$是$D$的聚点,如果存在常数$A$,对于任意给定正数$\varepsilon$,总存在正整数$\delta$,使得当点$P(x,\,y) \in D \cap \mathring{U}(P_0,\,\delta)$时,都有
                \[
                    |f(P)-A| = |f(x,\,y) - A| < \varepsilon
                \]
                成立,那么就称常数$A$为函数$f(x,\,y)$当$(x,\,y) \to (x_0,\,y_0)$时的极限(二重极限),记作
                \[
                    \lim_{(x,\,y) \to (x_0,\,y_0)}f(x,\,y) = A \quad \lor \quad \lim_{P \to P_0}f(P) = A
                \]
            \end{definition}

        \end{tcolorbox}


        % 莫兰迪棕色文本框美化
        \begin{tcolorbox}[
            title = \textbf{\textcolor{white}{Proposition}},
            colback=Salmon!20, 
            colframe=Salmon!90!Black]
            
            任意一点$P \in \R^2$与任意一个点集$E \subset \R^2$之间有以下三种关系的一种:
            \begin{itemize}[leftmargin=45pt]
                \item \textbf{内点}:如果存在点$P$的某个邻域$U(P)$,使得$U(P) \subset E$,那么称$P$为$E$的内点.
                \item \textbf{外点}:如果存在点$P$的某个邻域$U(P)$,使得$U(P) \cap E = \varnothing$,那么称$P$为$E$的外点.
                \item \textbf{边界点}:如果点$P$在任意邻域内既含有属于$E$的点,又含有不属于$E$的点,那么称$P$为$E$的边界点.
            \end{itemize}

        \end{tcolorbox}


        % 红色醒目文本框美化
        \begin{tcolorbox}[
            title=\textbf{\textcolor{white}{Theorem}}, 
            colback=red!5,
            colframe=red!75!black]
            红色醒目文本框美化


        \end{tcolorbox}


        % 绿色提示框
        \begin{tcolorbox}[
            title=\textbf{\textcolor{white}{Corollary}},
            colback=OliveGreen!10,
            colframe=Green!70]
            绿色提示框


        \end{tcolorbox}

        % 蓝色提示框
        \begin{tcolorbox}[
            title=\textbf{\textcolor{white}{Tips}},
            colback=JungleGreen!10!Cerulean!15,
            colframe=CornflowerBlue!60!Black]
            蓝色提示框

            
        \end{tcolorbox}

        
\end{document}


